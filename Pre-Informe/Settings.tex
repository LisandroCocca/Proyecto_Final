% Inclusion de paquetes
\usepackage[spanish]{babel}
\usepackage[utf8]{inputenc}
\usepackage{amsmath}
\usepackage{amsfonts}
\usepackage{amssymb}
\usepackage{enumitem}
\usepackage{multirow}
\usepackage{pdfpages}
\usepackage{float}
\usepackage{subfig}

\usepackage{tcolorbox}
\usepackage{sectsty}
\usepackage[hidelinks]{hyperref}
\usepackage{array}
\usepackage[font=small,labelfont=bf,tableposition=top]{caption}

\usepackage{booktabs}
\usepackage{titlesec}
\usepackage{color, soul, ulem}

\usepackage{fancyhdr}
\usepackage{lastpage}

\usepackage{sectsty}
\usepackage{ragged2e}
\usepackage{graphicx}
\usepackage{parskip}
\usepackage{gensymb}
\usepackage{array}

\usepackage[nottoc,notlot,notlof]{tocbibind}
\usepackage[left=2cm, right=1.50cm, top=2cm, bottom=2.00cm]{geometry}

% Profundidades
\setcounter{secnumdepth}{3}		% Numero hasta prof. 3
\setcounter{tocdepth}{3}		% Muestro en TOC hasta prof. 3
\AtBeginDocument{\renewcommand{\contentsname}{Tabla de Contenidos}}

% Definición de funciones
\newcommand\red{\color{red}}
\newcommand{\hsp}{\hspace{20pt}}

\usepackage{listings}
	\lstset{
		tabsize=4, language=Python
	}
	\lstset{
		morekeywords={cell,strcat}
	}

\definecolor{gray75}{gray}{0.75}
\definecolor{negro}{RGB}{0, 0, 0}
\definecolor{rojounm}{RGB}{201, 36, 70}
\definecolor{gris}{RGB}{226, 226, 226}

% Formateo del título chapter
\titleformat{\chapter}[hang]{\Huge\bfseries}{\thechapter\hsp\textcolor{gray75}{$\vert$}\hsp}{0pt}{\Huge\bfseries}



% Definicion de tamano de los titulos
\titleformat*{\section}{\LARGE\bfseries}
\titleformat*{\subsection}{\Large\bfseries}
\titleformat*{\subsubsection}{\large\bfseries}
\titleformat*{\paragraph}{\normalsize\bfseries}

% Definir el pagestyle estandar
\fancypagestyle{main}{%
	\fancyhf{}
	\fancyhead[R]{\leftmark}
	\fancyfoot[L]{COCCA, Lisandro; ROLLE, Guillermo}
	\fancyfoot[R]{\thepage\ de \pageref{LastPage}}
	\renewcommand{\headrulewidth}{0.4pt}
	\renewcommand{\footrulewidth}{0.4pt}
}

% Redefine chapter by adding fancy as the chapter title page page-style
\makeatletter
\let\stdchapter\chapter
\renewcommand*\chapter{%
	\@ifstar{\starchapter}{\@dblarg\nostarchapter}}
\newcommand*\starchapter[1]{%
	\stdchapter*{#1}
	\thispagestyle{empty}					% Afecta solo a la tabla de contenidos
	\markboth{\MakeUppercase{#1}}{}
}
\def\nostarchapter[#1]#2{%
	\stdchapter[{#1}]{#2}
	\thispagestyle{main}					% Paginas que arrancan un capitulo
}
\makeatother


%  Paquete 'todonotes' para notas de pendientes en el texto. Para eliminarlas agregar 'disable' en el header, como en las notas \thiswillnotshow (ver última en la lista)
\usepackage{lipsum}                     % Dummytext
\usepackage{xargs}                      % Use more than one optional parameter in a new commands
%\usepackage[pdftex,dvipsnames]{xcolor}  % Coloured text etc.
\usepackage[colorinlistoftodos,prependcaption,textsize=small]{todonotes}
\newcommandx{\unsure}[2][1=]{\todo[linecolor=red,backgroundcolor=red!25,bordercolor=red,#1]{#2}}
\newcommandx{\change}[2][1=]{\todo[linecolor=blue,backgroundcolor=blue!25,bordercolor=blue,#1]{#2}}
\newcommandx{\info}[2][1=]{\todo[linecolor=orange,backgroundcolor=orange!25,bordercolor=orange,#1]{#2}}
\newcommandx{\improvement}[2][1=]{\todo[linecolor=green,backgroundcolor=green!25,bordercolor=green,#1]{#2}}
\newcommandx{\thiswillnotshow}[2][1=]{\todo[disable,#1]{#2}}
%




